%%%%%%%%%%%%%%%%%%%%%%%%%%%%%%%%%%%%%%%%%%%%%%
%%                                          %%
%% USE THIS FILE TO SUBMIT YOUR SOLUTIONS   %%
%%                                          %%
%% You must have the usamts.tex file in     %%
%% the same directory as this file.         %%
%% You do NOT need to submit the usamts.tex %%
%% with your solutions.  You only need to   %%
%% submit this file and the PDF file with   %%
%% your solutions (and please attach any    %%
%% images you include)                      %%  
%%                                          %%
%% DO NOT ALTER THE FILE usamts.tex         %%
%%                                          %%
%% If you have any questions or problems    %%
%% using this file, or with LaTeX in        %%
%% general, please go to the LaTeX          %%
%% forum on the Art Of Problem Solving      %%
%% web site, and post your problem.         %%
%%                                          %%
%%%%%%%%%%%%%%%%%%%%%%%%%%%%%%%%%%%%%%%%%%%%%%

%%%%%%%%%%%%%%%%%%%%%%%%%%%%%%%%%%%%%%%%%%%%
%% DO NOT ALTER THE FOLLOWING LINES
\documentclass[11pt]{article}
\usepackage{amsmath,amssymb,amsthm}
\usepackage[pdftex]{graphicx}
\usepackage{fancyhdr}
\pagestyle{fancy}
\include{usamts}
%% DO NOT ALTER THE ABOVE LINES
%%%%%%%%%%%%%%%%%%%%%%%%%%%%%%%%%%%%%%%%%%%%


%% If you would like to use Asymptote within this document (which is optional), 
%% you can find out how at the following URL:
%%
%%   
%http://www.artofproblemsolving.com/Wiki/index.php/Asymptote:_Advanced_Configurat
%ion
%%
%% As explained there, you will want to uncomment the line below.  But be
%% sure to check the website because there are several other steps that must 
%% be followed.
 \usepackage{asymptote}
 \usepackage{amsmath}
 \usepackage{graphicx}
%% Enter your real name here
%% Example: \realname{David Patrick}
\realname{Benjamin Yin}

%% Enter your USAMTS username here
%% Example: \username{DPatrick}
%% IMPORTANT
%% If your username contains one of the following characters:
%%      # $ ~ _ ^ % { } &
%% then this character must be preceded by a backslash \
%% for example: if your user name is math_genius, then the line below should be
%% \username{math\_genius}

\username{ruiqiu2000}

%% Enter your USAMTS ID# here
%% Example: \usamtsid{9999}
\usamtsid{31020}

%% Make sure that the year and round number are correct.
%% Year 25 is the academic year 2013-14
\usamtsyear{25}
\usamtsround{1}

%%
%% DO NOT ALTER THE FOLLOWING LINE
\begin{document}
%% DO NOT ALTER THE PREVIOUS LINE
%%

%%%%%%%%%%%%%%%%%%%%%%%%%%%%%%%%%%%%%%%%%%%%%%%%%%%%%%%%
%%
%% All solutions go in between the \begin{solution} and
%% the \end{solution} corresponding to the problem
%% number.  
%%
%% For example, suppose Problem 2 is 
%% "Solve for x: x + 3 = 5".
%% Your solution would look like this:
%%
%% \begin{solution}{2}
%% If $x+3=5$, then subtract 3 from both
%% sides of the equation to get $x=5-3=2$, so the 
%% solution is $x=2$.
%% \end{solution}
%%
%%%%%%%%%%%%%%%%%%%%%%%%%%%%%%%%%%%%%%%%%%%%%%%%%%%%%%%%

\begin{solution}{1} 
\end{solution}

\begin{solution}{2}
We have our six numbers, $x, y, z, x-y, y-z, x-z$, all of which are prime.
Firstly, $x, y$, and $z$ must be distinct. If they are not distinct, then one 
of $x-y, y-z,$ or $x-z$ must not be zero, which is not a postive integer.
Let us then split this problem into two cases:

\textbf{1. None of $x, y, z$ are $2$}

If this is true, then $x, y$ and $z$ must be odd, as there are no even primes 
other than $2$.
However, this means that $x-y, y-z, x-z$ must be even, as they are the result 
of subtracting two odd numbers.
Therefore, $x-y$ and $y-z$ must both be $2$, since it is the only even prime.
However, if

\begin{center}
$\begin{cases} x-y = 2 \\ y - z = 2 \end{cases}$
\end{center}


Then

\begin{align*}
x - z &= (x - y) + (y - z)\\
&= 2 + 2\\
&= 4
\end{align*}

Therefore, $x-z$ is not prime, and therefore, if $x, y$ and $z$ are not $2$, 
then at least one of $x-y, y-z$ or $x-z$ is not prime.

\textbf{2. one of $x, y, z$ is $2$}
We know that $z$ must be $2$, as $x-z > 0$ and $y-z > 0$, $x > z$ and $y > z$, 
and $2$ is the smallest prime.
Because we know $z = 2$, $x-2$ and $y-2$ must be prime. Additionally, $x$ and 
$y$ are odd, as they are distinct from $z$ and $2$ is the only even prime. 
Therefore, $x-y$ is even, as it is the result of subtracting two odd numbers. 
The only solution for $x-y$ is therefore, $2$, and therefore, $x = y + 2$

Since $x = y + 2$ and $y-2$ must be prime, $x-4$ must be prime. 

$x, x-2$ and $x-4$ must all be prime. But this is impossible, as $x, x-2$ and 
$x-4$ are all distinct $\bmod{3}$, and therefore by pidgeonhole principle, one 
of them is divisible by $3$

Therefore, there are no triplets $(x, y, z)$ such that $x, y, z, x-y, y-z, x-z$ 
are all prime.

\end{solution}

\begin{solution}{3} %TODO: revise this.
a.
Let us first prove that after selecting any one person to be next in line, all 
of the people who are shorter than that person and not previously in the line
must also be in the line of the line to be in almost-perfect order.
Let us use proof by contradition: 
Let us assume that some person has been selected, let us call them X, such that 
they are taller than the person before them, Let us call them Y and that
there are people who are shorter than Y who are not currently in the line
Firstly, examine the shortest person not currently in the line. This person must 
come after the person 5 inches taller than them. Any other person is at least
10 inches greater in height, which is not allowed under almost-perfect order. 
However, this second person must come after the person 5 inches taller than him,
as everyone else must either come after this person, and one cannot be both 
simultaniously before and after a certain person in a line, 
or 10 or more inches taller than this person, which is not allowed under 
almost-perfect order. This reasoning repeats up until the person who is
5 inches taller than Y. This person cannot be placed after any person, as every 
other person must either be placed after this person by the reasoning above, or 
is already in line.
This means that this person cannot be placed in the line, and therefore, this 
line cannot be completed with everyone in the line and in almost-perfect
order. This means that after selecting any person and placing them in the line, 
unless the person who is 5 inches shorter than them is already in the line, we 
must select the person who is 5 inches shorter than them, and place them in the 
line, then apply this rule to the newly placed person, in essence placing anyone
who is shorter than the first selected person and not already in line after them in line.


The number of ways to count this can be found with recursion. In order to count the number of ways to place x people, all spaced by 5, into a line,
picking the shortest person reduces it to a line of 0 people, picking the second-shortest person reduces it to a line of one person, picking the third-shortest person reduces it to a line of 2 people, and so on until picking the tallest person results in a line of $x-1$
So if we define $f(x)$ as the number of ways to do this, we get

$
f(x) = \sum\limits_{i = 0}^{x-1} f(i)
$

Which, for 10 people, evaluates to 34
b.
The previous problem can be applied to this problem with the exception of the 
person with height 164. Let us handle the problem
by splitting it into two cases

\textbf{1. The first person of 164 and 165 selected is 164}

In this case, either the next-selected number is 165, in which case there is 1 
way to order the people greater
than 165, and there are  $2^{8}$ ways to order the people less than 165, From 
the previous problem. However, if the next number selected is 170, then 
we have 1 way to order the numbers above 170, and $2^{9}$ ways to order the 
numbers below 170. From the previous problem, if the next-selected number
is not one of these two numbers, we cannot complete the line in almost-perfect 
order.

\textbf{2. The first person of 164 and 165 selected is 165} 

In this case, either the next-selected number is 164, in which case there is 1 
way to order the people greater
than 164, and there are  $2^{8}$ ways to order the people less than 164, From 
the previous problem. However, if the next number selected is 170, then 
we have 1 way to order the numbers above 170, and $2^{9}$ ways to order the 
numbers below 170. From the previous problem, if the next-selected number
is not one of these two numbers, we cannot complete the line in almost-perfect 
order.

The grand total then is $2^{8} + 2^{9} + 2^{8} + 2^{9} = 1536$  %check this 
answer
\end{solution}
\begin{solution}{4}
%% Solution to Problem 4 goes here
\end{solution}

\begin{solution}{5}
Firstly, let us prove that all $a_i$, where $i\geq2$, is always equal to $a_{i+2}$. This can be done using induction:

Base Case: Firstly, $a_3 = a_5$, as $a_3 = a_{a_2} = a_5$. Since $a_5 = a_{3+2}$, we have our base case.

Inductive Step:
Assume we know that $a_{n-1} = a_{n+1}$. We also know that $a_{n+2} = a_{a_{n+1}} = a_{a_{n-1}} = a_n$.
Therefore, we know that $a_i = a_{i+2}$ for all $i\geq2$.

Now, we know that $a_{2014} = 2015$, which means all $a_n$ where $n$ is both even and greater than 3 is 2015, as $a_{2012} = a_{2014}, a_{2010} = a_{2012}$, etc. 
Additionally, since $a_{2016} = a_{a_{2015}}$, and$ a_{2016} = a_{2015}$, we have that $a_{a_{2015}} = 2015$
Also, let us show that since $a_i = a_{i+2}$, all $a_i$ where i is odd and greater than or equal to 2 are equal.

Let us split this into 4 cases:

Case 1: $a_{2015}$ is an even number greater than 3

In this case, $a_{a_{2015}} = a_i$, where i is even and greater than 3. This means that $i=2015$, as that is true of all $a_i$ satisfying the previous conditions, as stated above.

Case 2: $a_{2015}$ is an odd number greater than 2

If $a_{2015} = i$ where i is odd and greater than 2, then $a_{2016} = a_{a_{2015}} = a_i$. Additionally, since $i$ is odd and greater than 2, $a_i$

Case 3: $a_{2015} = 2$

If $a_{2015} = 2$, then $a_{2016} = a_{a_{2015}} = a_2 = 5$, but we know that $a_{2016} = 2015$, so $a_{2015}$ cannot equal 2.

Case 4: $a_{2015} = 1$

If $a_{2015} = 1$, then $a_{2016} = a_{a_{2015}} = a_1$. Therefore, since $a_{2016} = 2015$, we know that $a_1 = 2015$.
From that, $a_2 = a_{a_1} = a_{2015} = 1$, but we know that $a_2 = 5$. Therefore, $a_{2015}$ cannot equal 1. 

Therefore, $a_{2015}$ is any even number greater than 3

\end{solution}

%%
%% DO NOT ALTER THE FOLLOWING COMMAND
\end{document}
%% DO NOT ALTER THE PREVIOUS COMMAND
%%

%%%%%%%%%%%%%%%%%%%%%%%%%%%%%%%%%%%%%%%%%%%%%%
%%                                          %%
%% You may want to run LaTeX on this file   %%
%% to make sure that it compiles correctly  %%
%% before you submit it.                    %%
%%                                          %%
%% If you run LaTeX on this file before     %%
%% entering any solutions, the output file  %%
%% will be blank.  This is normal.          %%
%%                                          %%
%% Make sure that the file usamts.tex is    %%
%% in the same directory as this file.      %%
%% If the compiler hangs or asks for        %%
%% fancyhdr.sty when you compile, then go   %%
%% to the My Forms portion of My USAMTS     %%
%% at www.usamts.org to download the        %%
%% fancyhdr.sty file.                       %%
%%                                          %%
%% Questions?  Problems?  Go to the LaTeX   %%
%% forum on www.artofproblemsolving.com     %%
%%                                          %%
%%%%%%%%%%%%%%%%%%%%%%%%%%%%%%%%%%%%%%%%%%%%%%
